\documentclass[aspectratio=169]{beamer}

% --------------------
% Theme (clean + academic)
% --------------------
\usetheme{Madrid}
\usecolortheme{default}
\setbeamertemplate{navigation symbols}{}

% --------------------
% Packages
% --------------------
\usepackage{amsmath, amssymb, amsfonts}
\usepackage{mathtools}
\usepackage{bm}

% --------------------
% Shortcuts
% --------------------
\newcommand{\Om}{\Omega}
\newcommand{\M}{\mathcal{M}}
\newcommand{\Cz}{C_0(\Om)}
\newcommand{\ip}[2]{\left\langle #1,#2\right\rangle}
\newcommand{\norm}[1]{\left\lVert #1\right\rVert}

% --------------------
% Title info
% --------------------
\title[Elliptic Control with Measures]{Elliptic Optimal Control with Measure-Valued Controls}
\subtitle{Predual reformulation, sparsity, and semismooth Newton numerics}
\author{Anderson Singulani}
\institute{Seminar}
\date{January 21, 2026}

\begin{document}

% ============================================================
% 1. TITLE
% ============================================================
\begin{frame}
  \titlepage
\end{frame}

% ============================================================
% 2. ROADMAP
% ============================================================
\begin{frame}{Roadmap (25 min)}
\begin{itemize}
  \item Motivation: localized actuation and sparsity
  \item PDE with measure right-hand side: very weak solutions
  \item Primal problem in $\M(\Om)$: existence/uniqueness
  \item Fenchel duality $\Rightarrow$ predual (Hilbert space) with box constraints
  \item KKT + sparsity structure (active set)
  \item 1D gravity example: explicit Dirac solution + threshold
  \item Numerics: Moreau--Yosida + semismooth Newton / PDAS
  \item (Brief) alternative FEM route: nodal Dirac controls
\end{itemize}
\end{frame}

% ============================================================
% 3. MOTIVATION
% ============================================================
\begin{frame}{Motivation: why measure-valued controls?}
\begin{block}{Localized actuation}
Point sources/sinks, actuators, supports, injections are naturally \emph{low-dimensional} objects:
Dirac masses, sums of Diracs, line/surface measures.
\end{block}

\begin{block}{Sparsity promotion}
Replacing quadratic control cost by an $L^1$-type cost promotes sparsity.
In PDE settings, the measure space $\M(\Om)$ is the natural closure/relaxation.
\end{block}

\begin{alertblock}{Main challenge}
The problem is nonsmooth \emph{and} posed in a nonreflexive space $\M(\Om)$.
\end{alertblock}
\end{frame}

% ============================================================
% 4. PDE WITH MEASURE DATA
% ============================================================
\begin{frame}{Elliptic PDE with measure right-hand side}
We consider
\[
Ay = u \quad \text{in }\Om,\qquad y=0 \quad\text{on }\partial\Om,
\qquad u\in \M(\Om).
\]

\begin{block}{Why not the usual $H^1_0$ formulation?}
A general measure $u$ is \emph{not} in $H^{-1}(\Om)$, so
testing with $v\in H^1_0(\Om)$ is too strong.
\end{block}

\begin{block}{Very weak formulation (Dirichlet)}
Choose a test space $V \hookrightarrow C_0(\Om)$, e.g.
$V = H^2(\Om)\cap H^1_0(\Om)$ (in $n=2,3$).
Then $y\in L^1(\Om)$ solves
\[
\int_\Om y\, A^\ast \varphi\, dx = \int_\Om \varphi\, du
\qquad \forall \varphi\in V.
\]
\end{block}
\end{frame}

% ============================================================
% 5. REGULARITY / SOLUTION OPERATOR
% ============================================================
\begin{frame}{Regularity and the control-to-state map}
\begin{block}{Elliptic smoothing for measure data}
For $u\in \M(\Om)$, the very weak solution satisfies
\[
y \in W^{1,p}_0(\Om)\quad \text{for all } 1\le p < \frac{n}{n-1},
\qquad
\norm{y}_{W^{1,p}} \le C \norm{u}_{\M(\Om)}.
\]
\end{block}

\begin{block}{Control-to-state operator}
Define
\[
S:\M(\Om)\to L^2(\Om),\qquad Su = y(u),
\]
using Sobolev embedding $W^{1,p}_0(\Om)\hookrightarrow L^2(\Om)$ (for suitable $p$, $n\in\{2,3\}$).
\end{block}

\begin{alertblock}{Key takeaway}
Even if $u$ is singular (Dirac), the state $y$ is a genuine function usable in an $L^2$ tracking term.
\end{alertblock}
\end{frame}

% ============================================================
% 6. PRIMAL PROBLEM
% ============================================================
\begin{frame}{Primal sparse control problem in $\M(\Om)$}
Given desired state $z\in L^2(\Om)$ and $\alpha>0$:
\[
\min_{u\in \M(\Om)}\;
J(u):=\frac12 \norm{Su-z}_{L^2(\Om)}^2 + \alpha \norm{u}_{\M(\Om)}.
\]

\begin{block}{Interpretation}
\begin{itemize}
  \item $\frac12\norm{Su-z}^2$: match the target state
  \item $\alpha\norm{u}_{\M}$: pay for total control mass $\Rightarrow$ sparse actuation
\end{itemize}
\end{block}

\begin{alertblock}{Two difficulties}
\begin{itemize}
  \item Nonsmooth term $\norm{u}_\M$
  \item Nonreflexive space $\M(\Om)$
\end{itemize}
\end{alertblock}
\end{frame}

% ============================================================
% 7. EXISTENCE / UNIQUENESS
% ============================================================
\begin{frame}{Existence and uniqueness (direct method)}
\begin{block}{Compactness mechanism}
\begin{itemize}
  \item Minimizing sequence $(u_n)$ bounded in $\M(\Om)$
  \item Banach--Alaoglu: $u_n \stackrel{\ast}{\rightharpoonup} u^\ast$ in $\sigma(\M(\Om), C_0(\Om))$
  \item States $y_n=Su_n$ bounded in $W^{1,p}_0(\Om)$
  \item Rellich: $y_n \to y^\ast$ strongly in $L^2(\Om)$
\end{itemize}
\end{block}

\begin{block}{Lower semicontinuity}
\[
\norm{u^\ast}_\M \le \liminf_{n\to\infty}\norm{u_n}_\M,
\qquad
\norm{Su_n - z}_{L^2}^2 \to \norm{Su^\ast-z}_{L^2}^2.
\]
So $u^\ast$ is optimal.
\end{block}

\begin{alertblock}{Uniqueness}
Strict convexity of the $L^2$ tracking term (and injectivity of $S$) gives uniqueness of the minimizer.
\end{alertblock}
\end{frame}

% ============================================================
% 8. WHY PRE-DUAL?
% ============================================================
\begin{frame}{Why a predual formulation?}
\begin{block}{Goal}
Avoid discretizing measures directly and replace nonsmooth penalty by a simple constraint.
\end{block}

\begin{block}{Core idea (Fenchel duality)}
The conjugate of $\alpha\norm{\cdot}_\M$ is the indicator of a box in $C_0(\Om)$:
\[
(\alpha\norm{\cdot}_\M)^\ast(\varphi)
=
\begin{cases}
0, & \norm{\varphi}_\infty \le \alpha,\\
+\infty, & \text{otherwise}.
\end{cases}
\]
\end{block}

\begin{alertblock}{Computational win}
Measure norm $\Rightarrow$ \emph{box constraint} on a smooth variable.
\end{alertblock}
\end{frame}

% ============================================================
% 9. PRE-DUAL PROBLEM
% ============================================================
\begin{frame}{Predual problem (Hilbert space + box constraint)}
In the measure-control setting, the dual variable satisfies
\[
p \in H^2(\Om)\cap H^1_0(\Om) \hookrightarrow C_0(\Om).
\]

\begin{block}{Predual formulation}
\[
\min_{p\in H^2(\Om)\cap H^1_0(\Om)}
F(p):=\frac12 \norm{A^\ast p + z}_{L^2(\Om)}^2 - \frac12\norm{z}_{L^2(\Om)}^2
\quad \text{s.t.}\quad \norm{p}_\infty \le \alpha.
\]
\end{block}

\begin{alertblock}{Meaning}
All nonsmoothness is now in a \emph{simple} pointwise constraint.
\end{alertblock}
\end{frame}

% ============================================================
% 10. KKT CONDITIONS
% ============================================================
\begin{frame}{First-order optimality: KKT system}
Let $K=\{p\in H^2\cap H^1_0:\norm{p}_\infty\le \alpha\}$.

\begin{block}{Stationarity + normal cone}
There exists a multiplier $\lambda^\ast \in N_K(p^\ast)$ such that
\[
\nabla F(p^\ast) + \lambda^\ast = 0,
\qquad
\lambda^\ast \in N_K(p^\ast).
\]
\end{block}

\begin{block}{Identification of the primal control}
The multiplier corresponds to the primal solution:
\[
\lambda^\ast = -u^\ast.
\]
So, once $p^\ast$ is computed, the optimal measure control is recovered.
\end{block}
\end{frame}

% ============================================================
% 11. SPARSITY / ACTIVE SET
% ============================================================
\begin{frame}{Sparsity structure from complementarity}
\begin{block}{Saturation $\Rightarrow$ support of the measure}
From $p^\ast\in \alpha\,\partial\norm{u^\ast}_\M$ one gets:
\[
\norm{p^\ast}_\infty \le \alpha,
\qquad
\ip{u^\ast}{p^\ast} = \alpha \norm{u^\ast}_\M.
\]
\end{block}

\begin{alertblock}{Active set characterization}
The measure cannot charge the inactive region:
\[
|u^\ast|\big(\{x:\ |p^\ast(x)|<\alpha\}\big)=0.
\]
Hence $u^\ast$ is supported where $|p^\ast|=\alpha$.
\end{alertblock}

\begin{block}{Sign information}
\[
u^\ast \ge 0 \text{ on }\{p^\ast=\alpha\},
\qquad
u^\ast \le 0 \text{ on }\{p^\ast=-\alpha\}.
\]
\end{block}
\end{frame}

% ============================================================
% 12. EXAMPLE SETUP
% ============================================================
\begin{frame}{1D gravity example: tensioned string}
Let $\Om=(0,1)$ and consider a string under gravity + actuator force $u$:
\[
-y_{\text{phys}}'' = 1+u \quad \text{in }\mathcal{D}'(0,1),\qquad
y_{\text{phys}}(0)=y_{\text{phys}}(1)=0.
\]

\begin{block}{Shift to fit the abstract model}
Gravity-only state:
\[
-y_g''=1,\quad y_g(0)=y_g(1)=0
\quad\Rightarrow\quad
y_g(x)=\frac{x(1-x)}{2}.
\]
Let $y:=y_{\text{phys}}-y_g$, then
\[
-y''=u,\qquad z(x)=-y_g(x)=-\frac{x(1-x)}{2}.
\]
\end{block}
\end{frame}

% ============================================================
% 13. EXPLICIT SOLUTION
% ============================================================
\begin{frame}{Explicit optimal control: a single Dirac}
In this example one can construct a symmetric solution where the constraint is active only at $x=\tfrac12$.

\begin{block}{Result: optimal control is a Dirac}
\[
u^\ast = \left(48\alpha - \frac{5}{8}\right)\delta_{1/2}.
\]
\end{block}

\begin{block}{Interpretation}
The optimal actuator is concentrated exactly at the midpoint,
where the dual variable saturates the box constraint.
\end{block}

\begin{alertblock}{Message}
Sparse control is not just a slogan: it becomes a literal point actuator in 1D.
\end{alertblock}
\end{frame}

% ============================================================
% 14. THRESHOLD FOR NO CONTROL
% ============================================================
\begin{frame}{When is it optimal to apply \emph{no control}?}
The optimal control vanishes iff the unconstrained predual solution stays inside the box:
\[
u^\ast\equiv 0
\quad \Longleftrightarrow \quad
\norm{p_0}_\infty \le \alpha,
\]
where $p_0$ solves the unconstrained KKT equation.

\begin{block}{Gravity example: sharp threshold}
One computes
\[
\norm{p_0}_\infty = \frac{5}{384}.
\]
Hence
\[
u^\ast \equiv 0
\quad \Longleftrightarrow \quad
\alpha \ge \frac{5}{384}.
\]
\end{block}

\begin{alertblock}{Economic meaning}
Control appears only when its benefit exceeds its cost parameter $\alpha$.
\end{alertblock}
\end{frame}

% ============================================================
% 15. NUMERICS: REGULARIZATION
% ============================================================
\begin{frame}{Numerics: Moreau--Yosida regularization of the box}
\begin{block}{Problem}
The constraint $\norm{p}_\infty\le \alpha$ is nonsmooth (active-set structure).

\medskip
Regularize the indicator of the box constraint (Moreau--Yosida),
leading to a smooth approximation with parameter $\gamma>0$.
\end{block}

\begin{block}{Effect}
\begin{itemize}
  \item Allows Newton-type methods
  \item Retains active-set interpretation
  \item As $\gamma\to\infty$, solutions converge to the exact constrained problem
\end{itemize}
\end{block}
\end{frame}

% ============================================================
% 16. NUMERICS: SEMISMOOTH NEWTON / PDAS
% ============================================================
\begin{frame}{Semismooth Newton / primal-dual active set (PDAS)}
\begin{block}{High-level idea}
Rewrite the optimality system as a semismooth equation and apply Newton with generalized derivatives.
\end{block}

\begin{block}{Algorithmic structure}
\begin{itemize}
  \item Given $(p^k,\lambda^k)$, define active set via saturation of $|p^k|$
  \item Solve a linearized PDE system on inactive set
  \item Update and repeat until active set stabilizes
\end{itemize}
\end{block}

\begin{alertblock}{Performance}
Locally superlinear convergence (under standard assumptions).
\end{alertblock}
\end{frame}

% ============================================================
% 17. ALTERNATIVE ROUTE: FEM SEQUEL IDEA
% ============================================================
\begin{frame}{Alternative numerical route: variational discretization (FEM sequel)}
\begin{block}{Different philosophy}
Discretize only the \emph{state} equation by FEM, keep $u\in\M(\Om)$ continuous.
\end{block}

\begin{block}{Key theorem (informal)}
The discrete optimizer can be chosen as a \emph{finite combination of Diracs at mesh nodes}:
\[
u_h^\ast = \sum_{j=1}^{N(h)} \lambda_j \delta_{x_j}.
\]
\end{block}

\begin{alertblock}{Comparison}
\begin{itemize}
  \item Predual approach: smooth variable + box constraints + Newton
  \item FEM sequel: state discretization induces nodal sparsity automatically
\end{itemize}
\end{alertblock}
\end{frame}

% ============================================================
% 18. CONCLUSION
% ============================================================
\begin{frame}{Conclusion}
\begin{block}{Main messages}
\begin{itemize}
  \item Measure controls model localized actuation and yield sparse optimal solutions
  \item Very weak solutions provide well-posed PDE state equations for $u\in\M(\Om)$
  \item Fenchel duality gives a predual Hilbert-space formulation with $\norm{p}_\infty\le \alpha$
  \item KKT structure explains sparsity: $u^\ast$ lives on the active set $\{|p^\ast|=\alpha\}$
  \item 1D example: explicit Dirac optimal control + sharp no-control threshold
\end{itemize}
\end{block}

\begin{alertblock}{Takeaway}
Predual reformulation turns a difficult nonsmooth measure problem into a numerically friendly box-constrained PDE problem.
\end{alertblock}

\bigskip
\centering
\textbf{Thank you!}
\end{frame}

\end{document}
