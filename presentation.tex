\documentclass[aspectratio=169]{beamer}

% --------------------
% Theme (clean + academic)
% --------------------
\usetheme{Madrid}
\usecolortheme{default}
\setbeamertemplate{navigation symbols}{}

% --------------------
% Packages
% --------------------
\usepackage{amsmath, amssymb, amsfonts}
\usepackage{mathtools}
\usepackage{bm}

% --------------------
% Shortcuts
% --------------------
\newcommand{\Om}{\Omega}
\newcommand{\M}{\mathcal{M}}
\newcommand{\Cz}{C_0(\Om)}
\newcommand{\ip}[2]{\left\langle #1,#2\right\rangle}
\newcommand{\norm}[1]{\left\lVert #1\right\rVert}
\newcommand{\supp}{\operatorname{supp}}

% --------------------
% Title info
% --------------------
\title[Elliptic Control with Measures]{Elliptic Optimal Control with Measure-Valued Controls}
\subtitle{Predual reformulation, sparsity, and semismooth Newton numerics}
\author{Anderson Singulani}
\institute{Seminar}
\date{January 21, 2026}

\begin{document}

% ============================================================
% TITLE
% ============================================================
\begin{frame}
  \titlepage
\end{frame}

% ============================================================
% ROADMAP
% ============================================================
\begin{frame}{Roadmap}
\begin{itemize}
  \item Motivation: localized actuation and sparsity
  \item Setup PDE with measure data
  \item Optimization problem 
  \item Optimality conditions and sparsity structure 
  \item Numerics and regularization
  \item Experiments
  \item Conclusion
\end{itemize}
\end{frame}

% ============================================================
% MOTIVATION
% ============================================================
\begin{frame}{Motivation | why measure-valued controls?}

\begin{block}{Localized actuation in the real world}
Many control mechanisms are \emph{intrinsically concentrated} in space, e.g.
\begin{itemize}
  \item \textbf{Thermal control:} laser heating on a tiny spot, or micro-heaters on a chip (point-like heat sources).
  \item \textbf{Semiconductor devices:} carrier injection/extraction at small contacts (point-electrodes). 
  \item \textbf{Structural mechanics:} concentrated loads or actuators in beams/plates.
\end{itemize}
\end{block}

\begin{block}{Sparsity promotion}
Replacing a quadratic control cost by an $L^1$-type cost promotes sparse actuation.
In PDE settings, the measure space $\M(\Om)$ is the natural model.
\end{block}

\begin{alertblock}{Main challenge}
The problem is nonsmooth \emph{and} posed in the nonreflexive space $\M(\Om)$.
\end{alertblock}

\end{frame}

% ============================================================
% PDE WITH MEASURE DATA
% ============================================================
\begin{frame}{Setup | Elliptic PDE with measure data}
We consider
\[
Ay = u \quad \text{in }\Om,\qquad y=0 \quad\text{on }\partial\Om,
\qquad u\in \M(\Om).
\]

\begin{block}{Why not the usual $H^1_0$ formulation?}
A general measure $u$ is \emph{not} in $H^{-1}(\Om)$, so
testing with $v\in H^1_0(\Om)$ is not well-defined.
\end{block}

\begin{block}{Very weak formulation (Dirichlet)}
Choose a test space $V \hookrightarrow C_0(\Om)$, e.g.
$V = H^2(\Om)\cap H^1_0(\Om)$ (in $n=2,3$).
Then there exist $y\in L^1(\Om)$ that solves
\[
\int_\Om y\, A^\ast \varphi\, dx = \int_\Om \varphi\, du
\qquad \forall \varphi\in V.
\]
\end{block}
\end{frame}

% ============================================================
% REGULARITY / SOLUTION OPERATOR
% ============================================================
\begin{frame}{Setup | Regularity and the solution map}
\begin{block}{Elliptic smoothing for measure data}
For $u\in \M(\Om)$, the very weak solution satisfies
\[
y \in W^{1,p}_0(\Om)\quad \text{for all } 1\le p < \frac{n}{n-1},
\qquad
\norm{y}_{W^{1,p}} \le C \norm{u}_{\M(\Om)}.
\]
\end{block}

\begin{block}{Solution operator}
Define
\[
S:\M(\Om)\to L^2(\Om),\qquad Su = y(u),
\]
using Sobolev embedding $W^{1,p}_0(\Om)\hookrightarrow L^2(\Om)$ (for suitable $p$, $n\in\{2,3\}$).
\end{block}

\begin{alertblock}{Key takeaway}
Even if $u$ is singular (Dirac), the state $y$ is a genuine function usable in an $L^2$ tracking term.
\end{alertblock}
\end{frame}

% ============================================================
% PRIMAL PROBLEM
% ============================================================
\begin{frame}{Optimization problem | Primal sparse control problem in $\M(\Om)$}
Given desired state $z\in L^2(\Om)$ and $\alpha>0$:
\[
\min_{u\in \M(\Om)}\;
J(u):=\frac12 \norm{Su-z}_{L^2(\Om)}^2 + \alpha \norm{u}_{\M(\Om)}.
\]

\begin{block}{Interpretation}
\begin{itemize}
  \item $\frac12\norm{Su-z}^2$: match the target state
  \item $\alpha\norm{u}_{\M}$: pay for total control mass $\Rightarrow$ sparse actuation
\end{itemize}
\end{block}

\begin{alertblock}{Two difficulties}
\begin{itemize}
  \item Nonsmooth term $\norm{u}_\M$
  \item Nonreflexive space $\M(\Om)$
\end{itemize}
\end{alertblock}
\end{frame}

% ============================================================
% EXISTENCE / UNIQUENESS
% ============================================================
\begin{frame}{Optimization problem | Existence and uniqueness (direct method)}
\begin{block}{Compactness mechanism}
\begin{itemize}
  \item Minimizing sequence $(u_n)$ bounded in $\M(\Om)$
  \item Banach--Alaoglu: $u_n \stackrel{\ast}{\rightharpoonup} u^\ast$ in $\sigma(\M(\Om), C_0(\Om))$
  \item States $y_n=Su_n$ bounded in $W^{1,p}_0(\Om)$
  \item Rellich: $y_n \to y^\ast$ strongly in $L^2(\Om)$
\end{itemize}
\end{block}

\begin{block}{Lower semicontinuity}
\[
\norm{u^\ast}_\M \le \liminf_{n\to\infty}\norm{u_n}_\M,
\qquad
\norm{Su_n - z}_{L^2}^2 \to \norm{Su^\ast-z}_{L^2}^2.
\] So $u^\ast$ is optimal.
\end{block}

\begin{alertblock}{Uniqueness}
Strict convexity of the $L^2$ norm (and injectivity of $S$) gives uniqueness of the minimizer.
\end{alertblock}
\end{frame}

% ============================================================
% WHY PRE-DUAL?
% ============================================================
\begin{frame}{Optimization problem | Why a predual formulation?}
\begin{block}{Goal}
Avoid discretizing measures directly and replace nonsmooth penalty by a simple constraint.
\end{block}

\begin{block}{Core idea (Fenchel duality)}
The conjugate of $\alpha\norm{\cdot}_\M$ is the indicator of a box in $C_0(\Om)$:
\[
(\alpha\norm{\cdot}_\M)^\ast(\varphi)
=
\begin{cases}
0, & \norm{\varphi}_\infty \le \alpha,\\
+\infty, & \text{otherwise}.
\end{cases}
\]
\end{block}

\begin{alertblock}{Computational win}
Measure norm $\Rightarrow$ \emph{box constraint} on a smooth variable.
\end{alertblock}
\end{frame}

% ============================================================
% PRE-DUAL DERIVATION 1/2
% ============================================================
\begin{frame}{Optimization problem | Deriving the predual (Step 1)}
\begin{block}{Step 1: Reduced primal formulation}
Let $S:\mathcal M(\Omega)\to L^2(\Omega)$ denote the control-to-state map $Su=y(u)$.
The primal problem reads
\[
\min_{u\in\mathcal M(\Omega)}
\; \frac12\|Su-z\|_{L^2(\Omega)}^2+\alpha\|u\|_{\mathcal M(\Omega)}.
\]
\end{block}
\end{frame}


% ============================================================
% PRE-DUAL DERIVATION 2/2
% ============================================================
\begin{frame}{Optimization problem | Deriving the predual (Steps 2--5)}
\vspace{-1mm}

\begin{block}{Step 2--4: Idea (Fenchel duality)}
\begin{itemize}
  \item \textbf{Step 2:} Dualize the quadratic tracking term
  \[
  \frac12\|Su-z\|_{L^2}^2
  \;=\;
  \sup_{w\in L^2}\Bigl\{\langle Su-z,w\rangle-\frac12\|w\|_{L^2}^2\Bigr\}.
  \]
  \item \textbf{Step 3:} Move $S$ to the adjoint side: $\;\langle Su,w\rangle=\langle u,S^\ast w\rangle$,
  and set $p:=S^\ast w\in C_0(\Omega)$.
  \item \textbf{Step 4:} Eliminate $u$ using
  \[
  \sup_{u\in\mathcal M(\Omega)}\{\langle u,p\rangle-\alpha\|u\|_{\mathcal M}\}=0
  \ \Longleftrightarrow\ 
  \|p\|_\infty\le \alpha.
  \]
\end{itemize}
\end{block}

\begin{alertblock}{Step 5: Predual problem}
\[
\min_{p\in H^2(\Omega)\cap H_0^1(\Omega)}
\Bigl[\frac12\|A^\ast p+z\|_{L^2(\Omega)}^2-\frac12\|z\|_{L^2(\Omega)}^2\Bigr]
\quad\text{s.t.}\quad
\|p\|_\infty\le \alpha.
\]
\end{alertblock}
\end{frame}

% ============================================================
% PRE-DUAL PROBLEM
% ============================================================
\begin{frame}{Optimization problem | Predual problem}
In the measure-control setting, the dual variable satisfies
\[
p \in H^2(\Om)\cap H^1_0(\Om) \hookrightarrow C_0(\Om).
\]

\begin{block}{Predual formulation}
\[
\min_{p\in H^2(\Om)\cap H^1_0(\Om)}
F(p):=\frac12 \norm{A^\ast p + z}_{L^2(\Om)}^2 - \frac12\norm{z}_{L^2(\Om)}^2
\quad \text{s.t.}\quad \norm{p}_\infty \le \alpha.
\]
\end{block}

\begin{alertblock}{Meaning}
All nonsmoothness is now in a \emph{simple} pointwise constraint.
\end{alertblock}
\end{frame}

% ============================================================
% KKT DERIVATION 
% ============================================================
\begin{frame}{Optimality conditions | KKT conditions for the predual problem }
\vspace{-1mm}

\begin{block}{Step 1: Compute the derivative of $F$}
\[
\nabla F(p)=AA^\ast p + Az \in H_0^2(\Omega)^\ast.\]
\end{block}

\begin{block}{Step 2: KKT condition via subdifferentials}
Since $F$ is convex and Fr\'echet differentiable and $K$ is closed convex,
\[
0\in \partial(F+I_K)(p^\ast)=\nabla F(p^\ast)+\partial I_K(p^\ast).
\iff \exists\,\lambda^\ast\in N_K(p^\ast)\ \nabla F(p^\ast)+\lambda^\ast=0. \]

\end{block}

\begin{alertblock}{Step 3: KKT system (stationarity + variational inequality)}
Find $(p^\ast,\lambda^\ast)\in H_0^2(\Omega)\times H_0^2(\Omega)^\ast$ such that
\[
AA^\ast p^\ast + Az + \lambda^\ast = 0 \quad\text{in } H_0^2(\Omega)^\ast,
\]
\[
\langle \lambda^\ast,\,p-p^\ast\rangle_{H_0^2{}^\ast,H_0^2}\le 0
\quad \forall\,p\in H_0^2(\Omega)\ \text{with }\|p\|_{C_0}\le \alpha. \]
\end{alertblock}

\end{frame}


% ============================================================
% PRIMAL IDENTIFICATION + SPARSITY THEOREM
% ============================================================
\begin{frame}{Optimality conditions | Primal identification and sparsity }
\vspace{-1mm}

\begin{block}{Identify the primal solution $u^\ast$}
From the saddle-point / KKT system one obtains
\[
\lambda^\ast \in \partial I_K(p^\ast)=N_K(p^\ast)
\qquad\Longleftrightarrow\qquad
u^\ast := -\lambda^\ast \in \mathcal M(\Omega)
\]
\end{block}

\begin{block}{Sparsity / sign property}
For every test function $\psi\in C_c(\Omega)$ with $\psi\ge 0$:
\[
\langle u^\ast,\psi\rangle = 0
\quad \text{if }\supp(\psi)\subset\{|p^\ast|<\alpha\},
\]
\[
\langle u^\ast,\psi\rangle \ge 0
\quad \text{if }\supp(\psi)\subset\{p^\ast=\alpha\},
\qquad
\langle u^\ast,\psi\rangle \le 0
\quad \text{if }\supp(\psi)\subset\{p^\ast=-\alpha\}.
\]
\end{block}

\end{frame}

% ============================================================
% NUMERICS: REGULARIZATION
% ============================================================
\begin{frame}{Numerics and regularization | Moreau--Yosida regularization of the box}
\begin{block}{Problem}
The constraint $\norm{p}_\infty\le \alpha$ is nonsmooth (active-set structure).
\end{block}

\begin{block}{Effect}
Allows Newton-type methods
\end{block}

\begin{block}{Regularized predual problem $(P^\ast_{M,c})$}
For $c>0$, let $p_c\in H_0^2(\Omega)$ be the unique minimizer of
\[
\frac12\|A^\ast p+z\|_{L^2(\Omega)}^2-\frac12\|z\|_{L^2(\Omega)}^2
+\frac{1}{2c}\|\max(0,c(p-\alpha))\|_{L^2}^2
+\frac{1}{2c}\|\min(0,c(p+\alpha))\|_{L^2}^2,
\]
and define
\[
\lambda_c:=\max(0,c(p_c-\alpha))+\min(0,c(p_c+\alpha)).
\]
Then $(p_c,\lambda_c)$ solves $AA^\ast p_c+Az+\lambda_c=0$.
\end{block}

\end{frame}

% ============================================================
% CONVERGENCE OF THE REGULARIZATION 
% ============================================================
\begin{frame}{Numerics and regularization | Convergence of the Moreau--Yosida}
\small
\vspace{-1mm}



\begin{alertblock}{Theorem (Convergence as $c\to\infty$)}
Let $(p^\ast,\lambda^\ast)$ be the unique KKT solution of the unregularized box-constrained problem.
Then, as $c\to\infty$,
\[
p_c \to p^\ast \ \text{strongly in } H_0^2(\Omega),
\qquad
\lambda_c \rightharpoonup \lambda^\ast \ \text{weakly in } H_0^2(\Omega)^\ast.
\]
\end{alertblock}

\begin{block}{Proof idea }
\begin{enumerate}
\item (Key inequality) From the pointwise definition: $\;\langle \lambda_c,p_c\rangle_{L^2}\ge \frac1c\|\lambda_c\|_{L^2}^2$.
\item (Uniform bounds) Test $AA^\ast p_c+Az+\lambda_c=0$ with $p_c$ to bound $\|p_c\|_{H_0^2}$ and $\frac1c\|\lambda_c\|_{L^2}^2$.
\item (Subsequence limits) Extract $(p_c,\lambda_c)\rightharpoonup (\tilde p,\tilde\lambda)$ in $H_0^2\times H_0^2{}^\ast$ (Banach--Alaoglu).
\item (Feasibility) The penalties $\max(0,p_c-\alpha)$, $\min(0,p_c+\alpha)$ vanish in $L^2$ $\Rightarrow$ $|\tilde p|\le \alpha$ a.e.
\item (Strong conv.) Use weak l.s.c.\ to get $\|A^\ast p_c\|_{L^2}\to \|A^\ast\tilde p\|_{L^2}$, hence $p_c\to\tilde p$ in $H_0^2$.
\item (Limit KKT) Limit in the variational inequality to show $(\tilde p,\tilde\lambda)$ solves the unreg. KKT system.
\item (Uniqueness) Conclude $(\tilde p,\tilde\lambda)=(p^\ast,\lambda^\ast)$ and thus the whole family converges.
\end{enumerate}
\end{block}

\end{frame}


% ============================================================
% ALGORITHM: Semismooth Newton 
% ============================================================
\begin{frame}{Numerics and regularization | Algorithm: Semismooth Newton}
\small
\vspace{-1mm}

\begin{block}{Goal}
Solve the regularized optimality system $F(p)=0$ for $p_c\in H_0^2(\Omega)$
via an active-set (piecewise linear) Newton iteration.
\end{block}

\begin{enumerate}
\item Choose an initial guess $p^0\in H_0^2(\Omega)$ and set $k=0$.
\item Repeat:
\begin{enumerate}
\item[(i)] Define the active sets
\[
A_k^+ := \{x\in\Omega:\ p^k(x)>\alpha\},
\qquad
A_k^- := \{x\in\Omega:\ p^k(x)<-\alpha\},
\qquad
A_k := A_k^+\cup A_k^- .
\]
\item[(ii)] Compute $p^{k+1}\in H_0^2(\Omega)$ by solving the linear equation (weak form)
\[
\langle A^\ast p^{k+1},A^\ast v\rangle_{L^2}
+ c\langle p^{k+1}\chi_{A_k},v\rangle_{L^2}
=
-\langle z,A^\ast v\rangle_{L^2}
+ c\alpha\langle \chi_{A_k^+}-\chi_{A_k^-},v\rangle_{L^2}
\quad \forall v\in H_0^2(\Omega).
\]
\item[(iii)] Set $k\leftarrow k+1$.
\end{enumerate}
\item Stop if $A_k^+=A_{k-1}^+$ and $A_k^-=A_{k-1}^-$.
\end{enumerate}

\end{frame}

\begin{frame}{Experiments | Basic example}
    
\end{frame}

\begin{frame}{Experiments | Semiconductor example}
    
\end{frame}

\begin{frame}{Experiments | Semiconductor example}
    
\end{frame}

\begin{frame}{Experiments | Semiconductor example}
    
\end{frame}


% ============================================================
% 17. ALTERNATIVE ROUTE: FEM SEQUEL IDEA
% ============================================================
\begin{frame}{Alternative numerical route: variational discretization (FEM sequel)}
\begin{block}{Different philosophy}
Discretize only the \emph{state} equation by FEM, keep $u\in\M(\Om)$ continuous.
\end{block}

\begin{block}{Key theorem (informal)}
The discrete optimizer can be chosen as a \emph{finite combination of Diracs at mesh nodes}:
\[
u_h^\ast = \sum_{j=1}^{N(h)} \lambda_j \delta_{x_j}.
\]
\end{block}

\begin{alertblock}{Comparison}
\begin{itemize}
  \item Predual approach: smooth variable + box constraints + Newton
  \item FEM sequel: state discretization induces nodal sparsity automatically
\end{itemize}
\end{alertblock}
\end{frame}

% ============================================================
% 18. CONCLUSION
% ============================================================
\begin{frame}{Conclusion}
\begin{block}{Main messages}
\begin{itemize}
  \item Measure controls model localized actuation and yield sparse optimal solutions
  \item Very weak solutions provide well-posed PDE state equations for $u\in\M(\Om)$
  \item Fenchel duality gives a predual Hilbert-space formulation with $\norm{p}_\infty\le \alpha$
  \item KKT structure explains sparsity: $u^\ast$ lives on the active set $\{|p^\ast|=\alpha\}$ 
\end{itemize}
\end{block}

\begin{alertblock}{Takeaway}
Predual reformulation turns a difficult nonsmooth measure problem into a numerically friendly box-constrained PDE problem.
\end{alertblock}

\bigskip
\centering
\textbf{Thank you!}
\end{frame}

\end{document}
